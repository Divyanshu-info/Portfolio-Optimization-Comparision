\chapter{Introduction}
\label{introchap}

This chapter presents an overview about the context as a part of MiCas (Micro satellite
et Reseau de capteur sans fils) project in section 1.1. In section 1.2 the problems and
motivations are presented. Next, in section 1.3, due to the presented motivations, a
research work flow is introduced step by step. Finally, in section 1.4, the research
objectives are presented.

\section{Context}
The internship work is a part of MiCas project (Micro Satellite et Reseau de Capteur sans fils)
in LabSTICC, UBO. Micas project objective is develop and experiment solutions to coordinate
systems which will exchange data as well as control information between several distributed wireless
sensor networks (WSN) and several ground stations by using low cost micro satellites. The project
focus on a specification, verification and simulation of related situations : several distant WSN
have their gateways visited periodically by a mobile (satellite) on a static path.

\section{Problem/Motivation}
\textbf{Wireless sensor network and sensors} : Wireless sensor network (WSN) is a network which
implements many autonomous sensors to monitor physical or environmental conditions and passes
the collection data to a main location [5]. Inside the WSN, there are many devices, named sensor
can detect events or changes from environment and provides corresponding output. There are
several kinds of environment data that a sensor can detect: temperature, humidity, light...[6]. In
recent years, WSN is very important in environment monitoring, such as: air pollution, forest fire
detection, landslide detection... \\
\textbf{Distant WSN problem }: Most of the WSN projects is deployed in city (as example [7]), where
communication systems are abundant. Otherwise, there are several WSN projects focus to the
distant areas, such as shores, deserts, mountains, polar regions to monitor the environment changes.
With this case, it is admitted that the radio connection from the sensor nodes to gateway can be
lost or corrupted by effect of the environment. According to these problems, it is critical to propose
a (several) solution(s) to collect data periodically to ensure the well operation of WSN. \\
\textbf{Opportunities in Satellite cooperation} : Obviously, with the advances in Satellites or Unmanned
aerial vehicle (UAV) technologies, the solution to collect data periodically of WSN can
be achieved by implement Satellite or UAV. Generally, most of the satellite systems are industrial
systems and closed for the research purposes like MiCas project. Fortunately, the low earth orbit
satellites (LEO satellites) [8] like CubeSat are deployed in recent years can replace the industrial
satellites for the research projects i.e QB50 project, Outernet project [9, 10], Micas project because
at least two reasons: energy budget and solution cost. \\
\textbf{Cooperation problem} : However, besides the advantages in cooperation between WSNs and
CubeSat, there are several problems in data transaction:
\begin{itemize}
\item The correctly, reliable and adaptable protocols for data transaction between Satellite and
WSN.
\item An time event scheduler for WSN and satellite behaviors because they follow different sleeps
and sampling periods.
\item The limited buffering of the satellites to store and retrieve the data from ground stations. \\
\textbf{Research motivation} As the result, for the cooperation between CubeSat and WSN, it is
critical to build an application level simulation to develop and experiment the potentials as well
as the risks before real deploying in environment. 
\end{itemize}


\section{Objectives}
The objective of research is to approach the geo-modelisation, communication protocol development
and simulation for the interaction between CubeSat and WSN. This work also relates
to system investigations, based on the simulation of thousand of nodes in distant area, where
CubeSat visits them, collects data, connects to ground stations and controls sensing operations.
Consequently, during the internship, there are several objectives need to be achieved: \\
\textbf{Geo-modelisation on QuickMap} : To model the geo location of WSN, CubeSat trajectory and
the interactions between them. Moreover, in QuickMap, it allows to generate random network
topology in specified area, to record satellite path based on tracked longitude, latitude from GPredict
software (see chapter 2). \\
\textbf{Protocol developing in Occam environment} To develop the communication protocols, we
propose to use Occam structure because it uses micro threads and blocking channels ([11]) which
are suitable for mono and multi-processors simulation of WSN (as shown in [12]) (see chapter 3). \\
\textbf{Protocol implementing on GPGPU} : Meanwhile, the verified protocols will be implemented
on GPGPU to simulate the sensing activity, the network activities, and interactions with satellites
because GPGPU is attractive in synchronous message passing in WSN due to their Single Instruction
Multiple Data - SIMD like architecture, shared memory (see chapter 4). In addition, it also
useful to simulate the physical process due to the massive parallelism present in situations such as
flooding, fires, pollution [13].

\section{Research work flow}
According to the research objectives, the report will describe the work flow as below: \\
\textbf{Step 1} Using a map browser, QuickMap [14], to manage geo location of WSN and satellite trajectories
and to model the interaction between them. A satellite tracking software, GPredict
(see [15]), is used as an external process to passed the satellite paths and information to
QuickMap (chapter 2) \\
\textbf{Step 2} Using NetGen tool set [2] to generate the network topology from specification data of
WSN fields and Satellite on QuickMap. The network topology can be generated into Occam
structure or Compute Unified Device Architecture (CUDA) structure (chapter 2). \\
\textbf{Step 3} Developing and analyzing the distributed protocol algorithms for the cooperation between
Satellite and WSN based on generated Occam structure (chapter 3). \\
\textbf{Step 4} Using these proposed algorithms to develop a simulation with CUDA architecture on
General Purpose Graphic Processing Units (GPGPU) (chapter 4). Moreover, proposing a
specific debugger interface which allows to manage the simulation execution at high level.


\chapter{Discussions and conclusion}
In this chapter, the work is concluded and future plan is presented and the limitation of the work and possible future extensions are
described respectively.

\section{Final Conclusion} In this study, we use deep learning techniques to directly optimise the Sharpe ratio of a portfolio. By updating model parameters via gradient ascent, this pipeline avoids the limitations of conventional forecasting and enables us to better manage portfolio weights. BSE Sensex has been utilised as the benchmark portfolio in this study.
We contrast our approach with other well-known techniques, such as risk parity optimisation, maximum diversification, classical mean-variance optimisation reallocation techniques.
Our testing period, which runs from January 2000 to July 2022, encompasses the current COVID-19 situation.
The outcomes demonstrate that our model offers the greatest performance, and a thorough analysis of our model's performance in the actual scenario demonstrates the logic and viability of our approach.
\section{Contribution}
This study contributes to the field since there is no publication in the literary works that extensively and collectively covers the following topics: practical limitations, probabilistic and analytical approaches, different forms of performance evaluation, evaluation of strategy using back testing , and use of programming language and Deep Learning.\\\\
\section{Limitations}
There are two major limitations in this study that could be addressed in future research. \begin{itemize}
    \item Current Study Parameters can only include one Benchmark
    \item The study focused on only stock portfolios while there are many different types of assets whose value is appreciated over time.
    \item The stock data availability limits this study at the start of the benchmark period as stocks get listed at a different point in time
    
\end{itemize} 

\section{Future scope}
The next phase of this endeavour will include:
\begin{itemize}

\item Our goal is to examine portfolio performance using various objective functions. We may optimise the Sortino ratio or any other metric of the efficiency of the portfolio because to the flexible framework of our methodology.

\item We will Concentrate to build a portfolio utilising ETFs and market indexes rather than using individual assets. This significantly lowers the range of available assets, and these indexes have shown strong correlations.

\item Instead of just one, Our Technique can also include more market indices and sector-specific indices for Benchmarking Process
\end{itemize}
%\textcolor{white}{"}